\documentclass[a4paper, 12pt] {article}

\usepackage[T1]{fontenc}
\usepackage{paralist}
\usepackage{amsmath}
\usepackage{romannum}

\newcommand{\head}[1]{\textnormal{\textbf{#1}}}

\begin{document}

\title{ECON-102: Principals of Microeconomics}
\author{Luka Trikha}
\maketitle

\section{Unit 1: Fundamental Concepts}
\subsection{Section 1: Economics}
In general terms, economics is defined as the study of how we can best increase
a nation's standard of living and citizens' happiness with the resources that we
have available to us.\\[2mm]
Standards of living include:
\begin{itemize}
    \item cars
    \item houses
    \item leisure time
    \item access to health care
    \item cleaner air
\end{itemize}

\subsubsection{Marginal Benefit \& Marginal Cost}
Marginal benefit and marginal cost can be though of as a positive cause-and-effect
in a business environment, with the benefit being the effect and cost being the
cause. When your marginal benefit is greater the marginal cost, the more likely
a positive investment is at play. For example, you may buy an expensive car for
your long commute, but it has the best MPG in the current car market and is 
heavily relaiable (maringal benefit)--potentially outweighing the initial cost
(marginal cost).

\subsubsection{Difference between Macro- \& Micro- economics}
Macroeconomics focuses on the wider concepts that play a role on the entire
economy. Components of this include:
\begin{itemize}
    \item national unemployment rate
    \item inflation rate
    \item interest rate
    \item federal government budgets \& fiscal policies
    \item economic growth
    \item Federal Reserve System \& monetary policy
    \item foreign exchange rates
    \item balance of payments
\end{itemize}
Microeconomics deals with the smaller concepts of an economy such as:
\begin{itemize}
    \item supply and demand of individual goods and services
    \item price elasticity (sensitivity) of goods and services in demand
    \item production
    \item cost functions
    \item business behavior and profit maximization
    \item income inequality \& distribution
    \item effects of protectionism (tarrifs, quotas, trade restrictions, etc.)
\end{itemize}
If macroeconomics is studying a forest, microeconomics is studying the
individual trees.

\subsection{Section 2: The Production Possibilities Curve}
\subsubsection{Production Choices}
\end{document}
