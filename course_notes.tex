\documentclass[a4paper, 12pt] {article}

\usepackage[T1]{fontenc}
\usepackage{paralist}
\usepackage{amsmath}
\usepackage{romannum}
\usepackage{graphicx}

\graphicspath{{./images/}}
\newcommand{\head}[1]{\textnormal{\textbf{#1}}}

\begin{document}

\title{ECON-102: Principals of Microeconomics}
\author{Luka Trikha}
\maketitle

\section{Unit 1: Fundamental Concepts}
\subsection{Section 1: Economics}
In general terms, economics is defined as the study of how we can best increase
a nation's standard of living and citizens' happiness with the resources that we
have available to us.\\[2mm]
Standards of living include:
\begin{itemize}
    \item cars
    \item houses
    \item leisure time
    \item access to health care
    \item cleaner air
\end{itemize}

\subsubsection{Marginal Benefit \& Marginal Cost}
Marginal benefit and marginal cost can be though of as a positive cause-and-effect
in a business environment, with the benefit being the effect and cost being the
cause. When your marginal benefit is greater the marginal cost, the more likely
a positive investment is at play. For example, you may buy an expensive car for
your long commute, but it has the best MPG in the current car market and is 
heavily relaiable (maringal benefit)--potentially outweighing the initial cost
(marginal cost).

\subsubsection{Difference between Macro- \& Micro- economics}
Macroeconomics focuses on the wider concepts that play a role on the entire
economy. Components of this include:
\begin{itemize}
    \item national unemployment rate
    \item inflation rate
    \item interest rate
    \item federal government budgets \& fiscal policies
    \item economic growth
    \item Federal Reserve System \& monetary policy
    \item foreign exchange rates
    \item balance of payments
\end{itemize}
Microeconomics deals with the smaller concepts of an economy such as:
\begin{itemize}
    \item supply and demand of individual goods and services
    \item price elasticity (sensitivity) of goods and services in demand
    \item production
    \item cost functions
    \item business behavior and profit maximization
    \item income inequality \& distribution
    \item effects of protectionism (tarrifs, quotas, trade restrictions, etc.)
\end{itemize}
If macroeconomics is studying a forest, microeconomics is studying the
individual trees.

\subsection{Section 2: The Production Possibilities Curve}
\subsubsection{Production Choices}
Production choices are the idea that if you have limited resources to produces
various products, you want to optimize the resources at hand so that you can make
the most of the available resources, not underuse, and not over-promise a production
value that is not achievable.

\subsubsection{Points on the Curve and Trade-Offs}
In a given graph, any values that lie on the curve means that the operating cost
of the products are being used as efficiently as possible. The idea is that the
output cannot increase if it is limited by a constant resource and technology.
Scarcity talks about the limited resources at hand--which directly correlates
with the Production Possibility Curve. If a value lands on the curve, increasing
the production of one good/category will be at the expense of other goods/categories.
Points E, C, B, A, and D depicted in firgure \ref{fig:GnR1} represents the most
optimized products that can be produced with resources at hand. It also shows
varying priorities for both Guns and Roses productions.

Any points that fall inside the curve (to the left of the curve, i.e point G in
figure \ref{fig:GnR1}) shows an inefficent use of resources to produce products.
Some reasons for this could be using fewer than the available resources
(unemployment), or using all resources but inefficiently (underemployment).

Points that fall outside the curve (to the right of the curve, i.e point G in
figure \ref{fig:GnR1}) shows a combination that cannot be achieved with the
available resources. This value does not mean point F will never be achievable--
the economy may grow and F may fall on or inside the Possibility curve, but at
the current analysis of the economy, it will not be possible.
Increases in technology and/or resources can help contribute to the growth of
the Production Probability Curve, which can help reach point F in the future.

\begin{figure}[h]
    \centering
    \includegraphics[width=7.5cm, height=5cm]{Production_Possibility_Curve.jpg}
    \caption{Example of a Possibility Curve of Guns and Roses production}
    \label{fig:GnR1}
\end{figure}

\subsection{Section 3: Economic Growth}
Economic growth occurs when the economy realizes greater production levels.
Essentailly, when either the number of resources increase, or the way we use
resources becomes more efficent, is the only time the curve can shift outwards.
In short, economic growth is made possible by advances in technology and/or
increase in resources.

\begin{figure}[h]
    \centering
    \includegraphics[width=7.5cm, height=5cm]{economic_growth_graph.jpg}
    \caption{Example of how economic growth now reaches point F}
    \label{fig:GnR2}
\end{figure}

\subsubsection{Increase in Capital Goods}
If a country is producing at full employment, more capital goods can be produced
only inf the country produces fewer consumption goods. A few ways governments
can encourage more production of capital goods can be through tax breaks for the
production of capital goods, or increasing taxes on the production/sale of
non-capital (consumption) goods.

\subsubsection{Advances in Technology}
Advancements in technology that contribute to economic growth are usually due to
entrepreneurs who have incenteives to produce more efficiently and lower their
costs. When this model is successful, this usually drives the entrepreneur to 
continue to improve their models to become more efficent with both the work/effort
needed, and the money saved. Governments that allow entrepreneurs to keep most of
their profits and tax them less has been shown to produce greater rates of
technological growth. In addition to new technology, the more human technological
advancements made (greater education, training, skills, etc), the higher the 
production probability curve also grows.

\subsubsection{Economic Growth and Economic Systems}
There are various factors that can lead a country to economic growth and downfalls.
For example, in a capitalist country, having a government that supports just 
reward systems (taxes and regulations that reward work and entrepreneurship), 
just legal system, infrastructure, national security, and protection of individual
property rights can all lead to great economic growth. Also, political incentives
can also lead to economic growth. For example, India's switch to international 
trade in the 90's has led to greater opportunities, and the same for China in
the 80's when they adopted the free market elements.

Countries that practice communistic or command economy policies have seen 
signifigantly less economic growth due to the sheer control the government has
over resources and entrepreneural incentives.

In third world countries, instability with governments, corruption, civil strife,
national security, and uncertainbilities make is extremly difficult to have a 
steady, growing economy.

\subsubsection{Conditions for Economic Growth}
Countries with the highest per capita earnings are characterized by all or most
of the following:

\begin{enumerate}
    \item \textbf{Strong private property rights.}

\end{enumerate}
\end{document}
